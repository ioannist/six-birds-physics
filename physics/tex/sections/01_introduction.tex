\section{Introduction}
\label{sec:introduction}

To ``become a stone'' in physics is to pass from a detailed microdescription to a stable macro layer that can be treated as its own theory: the high-frequency degrees of freedom are discarded, the retained macro information becomes actionable, and the resulting description behaves coherently on the scale of interest. This is precisely the closure move.

Many physics ``theory transitions'' are implemented as scale choices: retain some macro information, discard the rest, and add a closure rule to make the reduced description usable.

What is often missing is a \emph{checkable, cross-domain way} to compare such closures: when does a proposed macro layer behave coherently, when does it preserve a chosen audit (data-processing), and when does it necessarily induce a structured mismatch term?

\emph{Six Birds Theory} (SBT) is a closure-based emergence calculus introduced in~\citep{SixBirdsFramework}; readers unfamiliar with its primitives will find all required definitions self-contained in Section~\ref{sec:dictionary}. This is a \emph{research article}: it presents original computational results, mechanized proofs, and reproducible experiments that instantiate the SBT closure-audit language in four physics domains.

This paper is a follow-up instantiation of the Six-Birds dictionary in several physics settings. We treat a layer as a choice of lens $\Qf$ (what is retained), completion $\Uf$ (how discarded information is completed), and a timescale evolution operator $T_\tau$, inducing a packaging operator $\Etauf=\Uf\circ\Qf\circ T_\tau$. The same three diagnostics recur across domains: closure coherence (idempotence), audit monotonicity, and route mismatch (noncommutation).

\paragraph{Contributions (three-layer stack).}
We (A) introduce a reusable closure-layer template aligned with the Six-Birds certificate loop, (B) operationalize it as a reproducible workflow that exports figures/tables from deterministic scripts and records Lean-certified backbone lemmas, and (C) demonstrate controlled null-vs-intervention separations in four instantiations (quantum dephasing, BGK-like moment closure, filtering/LES on Burgers, and nonlinear averaging/backreaction model).

\paragraph{Computational contributions.}
From a computer science and information theory perspective, this paper delivers:
\begin{itemize}[noitemsep,topsep=2pt]
  \item Mechanized structural lemmas in Lean~4: section-based closure idempotence, factorization implies commutation, total-variation contraction under deterministic pushforward, and definability counting.
  \item A reproducible Python simulation suite with deterministic seeding and hash-traceable artifacts across four physics domains.
  \item Information-theoretic diagnostics (idempotence defect, route mismatch via trace distance and $L^2$ norms, audit monotonicity via quantum relative-entropy DPI) as operational certificates for model-reduction quality.
  \item Explicit failure-mode documentation showing when and why closures degrade under misspecification.
\end{itemize}

\paragraph{Code availability.}
The repository is available at:
\begin{itemize}[noitemsep,topsep=2pt]
\item \url{https://github.com/ioannist/six-birds-physics}
\item Archived snapshot: \href{https://doi.org/10.5281/zenodo.18451876}{10.5281/zenodo.18451876}
\end{itemize}

\paragraph{Paper map.}
Section~\ref{sec:dictionary} fixes the dictionary and notation; Section~\ref{sec:layers-closures} states the thesis and the cross-domain invariants.
Sections~\ref{sec:quantum-classical}--\ref{sec:gravity} present instantiations.
Appendix~\ref{app:lean} lists Lean-certified lemmas used as algebraic anchors, and Appendix~\ref{app:sims} records the discrepancy metrics and default run settings for the numerical certificates.
