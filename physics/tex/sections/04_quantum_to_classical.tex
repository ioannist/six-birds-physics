\section{Quantum $\to$ classical: closure as dephasing}
\label{sec:quantum-classical}

This section instantiates the Six-Birds layer template in finite-dimensional quantum mechanics and one of its simplest ``classicalization'' procedures: dephasing in a preferred basis. The point is not interpretational; it is structural. Dephasing is a concrete completion/closure choice that is \emph{exactly idempotent}, satisfies a strong audit monotonicity principle (data processing for quantum relative entropy), and typically does \emph{not} commute with coherent (unitary) evolution---producing a route mismatch that is easy to visualize.

\subsection{Micro state, lens, and closure}
We take the micro state space to be density matrices $\rho$ on a $d$-dimensional Hilbert space. The analog of a lens is ``retain only diagonal information in a chosen basis,'' i.e.\ keep the classical probability vector given by the diagonal entries. The corresponding closure is the dephasing channel
\[
  \mathcal{C}(\rho) := \Delta(\rho),
\]
where $\Delta$ is the completely positive trace-preserving (CPTP) map that zeroes all off-diagonal entries in the chosen basis \citep{nielsenchuang2010,zurek2003,schlosshauer2007}.

To make the correspondence with the $(\Qf,\Uf,E_f)$ template explicit, let $Q$ map $\rho$ to the classical distribution $p$ on basis states given by $p_i=\rho_{ii}$, and let $U$ embed such a distribution as the diagonal density matrix $\mathrm{diag}(p)$. Then $\mathcal{C}=U\circ Q$ is exactly the induced closure.

Two closure properties are immediate.
\begin{itemize}
  \item \textbf{Exact idempotence.} Dephasing is a projection: $\mathcal{C}(\mathcal{C}(\rho))=\mathcal{C}(\rho)$ for all $\rho$. Figure~\ref{fig:quantum-closure-idempotence} verifies this numerically and also shows that ``partial'' dephasing (a convex combination of identity with $\Delta$) interpolates between non-idempotent and idempotent behavior.

  \item \textbf{A canonical completion family.} The image of $\mathcal{C}$ is the set of diagonal density matrices, i.e.\ classical distributions embedded as commuting states. In Six-Birds language, this is a distinguished macro-consistent family selected by the closure.
\end{itemize}

\subsection{Audit monotonicity: quantum DPI (numerical certificate)}
Quantum relative entropy
\[
  S(\rho\|\sigma)
\]
is the canonical audit for distinguishability of quantum states. A foundational monotonicity principle states that for any CPTP map $\Phi$,
\[
  S(\rho\|\sigma) \ge S(\Phi(\rho)\|\Phi(\sigma)).
\]
This is the quantum data processing inequality (DPI): applying a channel cannot increase distinguishability.
We cite standard sources for DPI and its variants \citep{lindblad1975,petz1986,wilde2017}.
We stress that DPI itself is a standard theorem; our numerical suite is a regression test for the correctness and numerical stability of our implementations of quantum relative entropy and channel application (Appendix~\ref{app:sims} records the default trial counts and tolerances).

Our repo includes a numerical audit suite that samples random density matrices and random CPTP channels (mixtures of dephasing, depolarizing, and random Kraus families), and records the differences
\[
  S(\rho\|\sigma) - S(\Phi(\rho)\|\Phi(\sigma)).
\]
Figure~\ref{fig:quantum-dpi-hist} shows the distribution of these differences across random trials, and Table~\ref{tab:quantum-dpi-summary} summarizes the experiment across dimensions.

\subsection{Route mismatch: coherent evolution vs closure}
Let the micro dynamics be unitary evolution generated by a Hamiltonian $H$:
\[
  U_t(\rho) := e^{-iHt}\,\rho\,e^{iHt}.
\]
In general, dephasing and unitary evolution do \emph{not} commute:
\[
  \mathcal{C}(U_t(\rho)) \neq U_t(\mathcal{C}(\rho)).
\]
This is exactly the Six-Birds ``route mismatch'' diagnostic: evolve then close versus close then evolve.

Figure~\ref{fig:quantum-route-mismatch} plots the \emph{trace distance} between the two routes as a function of time. The behavior is interpretable:
\begin{itemize}
  \item If the Hamiltonian is diagonal in the dephasing basis, then $U_t$ preserves diagonality and the routes commute (mismatch $\approx 0$).
  \item If $H$ mixes the basis, coherent evolution continually regenerates off-diagonal components which are then removed by dephasing, and the mismatch is generically nonzero.
\end{itemize}

\subsection{What this instantiation teaches about ``layers''}
Dephasing provides a clean example of a physics layer as a closure:
\begin{enumerate}
  \item The closure is \emph{exactly idempotent} (a perfect projection onto a macro-consistent family).
  \item The audit (quantum relative entropy) is \emph{monotone} under channels (a strong form of audit monotonicity).
  \item Route mismatch is generically nonzero, and its magnitude is a concrete diagnostic of incompatibility between the chosen closure and the chosen dynamics.
\end{enumerate}
This structure mirrors the classical, kinetic, fluid, and gravity examples. The lens and completion define what the ``layer'' keeps fixed; audits certify what cannot be gained by coarse-graining; route mismatch exposes the effective terms needed when the layer is not dynamically closed.

\begin{figure}[t]
  \centering
  \includegraphics[width=0.85\linewidth]{\figQuantumDpiHist}
  \caption{Quantum audit monotonicity regression: histogram of $\Delta = S(\rho\|\sigma)-S(\Phi(\rho)\|\Phi(\sigma))$ over random state pairs and CPTP channels; all trials satisfy $\Delta \ge -\mathrm{tol}$ (tolerance set in the suite).}
  \label{fig:quantum-dpi-hist}
\end{figure}

\begin{table}[h]
  \centering
  \caption{Quantum DPI summary.}
  \label{tab:quantum-dpi-summary}
  \begin{tabular}{lllll}
    \toprule
    d & trials & min diff & mean diff & violations \\ 
    \midrule
    2 & 300 & 0.00672 & 0.683 & 0 \\ 
    3 & 300 & 0.0352 & 0.932 & 0 \\ 
    4 & 300 & 0.134 & 1.05 & 0 \\ 
    6 & 300 & 0.293 & 1.06 & 0 \\ 
    \bottomrule
  \end{tabular}
\end{table}


\begin{figure}[t]
  \centering
  \includegraphics[width=0.85\linewidth]{\figQuantumClosureIdempotence}
  \caption{Dephasing closure coherence: trace-distance idempotence defect $\tfrac12\|\mathcal{C}_\lambda(\mathcal{C}_\lambda(\rho))-\mathcal{C}_\lambda(\rho)\|_1$ decreases as $\lambda\to 1$ and is numerically $\approx 0$ at $\lambda=1$ (exact projection).}
  \label{fig:quantum-closure-idempotence}
\end{figure}

\begin{figure}[t]
  \centering
  \includegraphics[width=0.85\linewidth]{\figQuantumRouteMismatch}
  \caption{Route mismatch between unitary evolution and closure: trace distance $\tfrac12\|\mathcal{C}_\lambda(U_t(\rho)) - U_t(\mathcal{C}_\lambda(\rho))\|_1$ versus time; mismatch is zero when $H$ is diagonal in the dephasing basis and nonzero otherwise.}
  \label{fig:quantum-route-mismatch}
\end{figure}

\FloatBarrier
