\section{Filtering/LES: route mismatch and the subgrid rewrite term}
\label{sec:les}

Large-eddy simulation (LES) is a standard setting where a coarse description is obtained by \emph{filtering} a fine-scale field, and where noncommutation between ``coarse then evolve'' and ``evolve then coarse'' is not a pathology but the central mechanism that produces a closure term \citep{germano1992,pope2000,sagaut2006,leonard1974,smagorinsky1963}. This section instantiates the Six-Birds route-mismatch diagnostic in a minimal nonlinear PDE toy: 1D viscous Burgers on a periodic grid.

\subsection{Lens as a spatial filter}
Let $u(x,t)$ be a fine-scale field (the micro description). Fix a filter kernel $G_\sigma$ (e.g.\ a Gaussian or box filter) with width $\sigma>0$, and define the filtered (macro) field
\[
  \bar{u}(x,t) := (G_\sigma * u)(x,t).
\]
This filtering operator is the lens $\Lens$ in Six-Birds language: it retains large-scale content and discards fine-scale fluctuations.
Unlike the finite and quantum instantiations, filtering is not (and is not intended to be) invertible: there is no canonical completion $\Uf$ that reconstructs the full field from $\bar{u}$. In this section we therefore focus on the lens-only commutator $Q\circ T_\tau - T_\tau\circ Q$ and the induced rewrite term, rather than on an idempotent closure $E=\Uf\circ\Qf$.

\subsection{Nonlinearity forces route mismatch}
Consider viscous Burgers,
\[
  \partial_t u + \partial_x\!\left(\tfrac{1}{2}u^2\right) = \nu\,\partial_{xx}u.
\]
Filtering is linear, so it commutes with the viscous term. However, it does \emph{not} commute with the nonlinear flux:
\[
  \overline{u^2} \neq (\bar{u})^2 \quad \text{in general}.
\]
For a translation-invariant convolution filter on a periodic domain (our setting), filtering commutes with spatial derivatives, so the filtered equation can be written exactly as
\[
  \partial_t \bar u + \partial_x\!\left(\tfrac{1}{2}(\bar u)^2\right)
  = \nu\,\partial_{xx}\bar u - \partial_x\!\left(\tfrac{1}{2}\tau_{\mathrm{sgs}}\right),
  \qquad
  \tau_{\mathrm{sgs}} := \overline{u^2} - (\bar u)^2.
\]
This is exactly the Six-Birds route mismatch at the PDE level: filtering after a nonlinear evolution step is not the same as evolving the filtered field under the same nonlinear rule.

Operationally, we quantify the mismatch using a discrete time-stepper $T_\tau$ and compare the two routes:
\[
  \text{(A)}\quad \overline{T_\tau(u)} \qquad \text{versus}\qquad
  \text{(B)}\quad T_\tau(\bar{u}).
\]
Their difference is the route mismatch. In our toy experiment, the mismatch magnitude increases with filter width $\sigma$ (Figure~\ref{fig:les-route-mismatch} and Table~\ref{tab:les-route-mismatch}), reflecting that stronger coarse-graining discards more of the nonlinear interaction structure.
We report mismatch magnitude using a root-mean-square ($L^2$) norm over the spatial grid.
\paragraph{Implementation note.}
Our toy uses a pseudo-spectral derivative with RK4 time stepping in a viscous regime (fixed $\nu$ and a conservative $\Delta t$) over $t\in[0,1]$, and the code explicitly checks for NaNs/infs.
These settings keep the solution smooth so that the reported mismatch primarily reflects noncommutation rather than numerical blowup.

\subsection{The subgrid rewrite term}
Filtering Burgers yields an exact equation for $\bar{u}$ with an additional term. A canonical way to express it is through the subgrid stress (or subgrid flux) identity
\[
  \tau_{\mathrm{sgs}}(x,t) := \overline{u^2}(x,t) - \bigl(\bar{u}(x,t)\bigr)^2.
\]
This quantity is identically zero for linear dynamics and generically nonzero for nonlinear dynamics. In Six-Birds terms, $\tau_{\mathrm{sgs}}$ is a \emph{rewrite term}: it is the structured correction required to make the filtered (macro) evolution match the filtered image of the micro evolution.

Our toy computation evaluates $\tau_{\mathrm{sgs}}$ and reports its magnitude over time as a function of $\sigma$. The results show a clear scaling with filter width: larger $\sigma$ produces larger subgrid terms (Figure~\ref{fig:les-subgrid-term} and Table~\ref{tab:les-subgrid-term}).

\subsection{Takeaway}
LES exemplifies the ``layer as closure'' thesis in its most concrete form:
\begin{itemize}
  \item the lens is explicit (filtering),
  \item route mismatch is unavoidable for nonlinear dynamics and is measurable,
  \item the mismatch is not arbitrary noise; it induces an exact algebraic correction term, $\tau_{\mathrm{sgs}}$, which is the object that any practical closure model must approximate.
\end{itemize}
In this way, LES provides a physics-native illustration of the Six-Birds commutator picture: a macro layer becomes dynamically meaningful only after accounting for the structured mismatch between coarse-graining and nonlinear evolution.

\begin{table}[t]
\centering
\setlength{\tabcolsep}{4pt}
\renewcommand{\arraystretch}{0.9}
\begin{minipage}[t]{0.48\linewidth}
  \centering
  \captionof{table}{LES route mismatch summary.}
  \label{tab:les-route-mismatch}
  \small
\begin{tabular}{ll}
  \toprule
  sigma & max mismatch \\ 
  \midrule
  0 & 0 \\ 
  0.0491 & 0.00132 \\ 
  0.0982 & 0.00495 \\ 
  0.196 & 0.0157 \\ 
  0.393 & 0.0332 \\ 
  \bottomrule
\end{tabular}

\end{minipage}\hfill
\begin{minipage}[t]{0.48\linewidth}
  \centering
  \captionof{table}{LES subgrid term magnitude.}
  \label{tab:les-subgrid-term}
  \small
\begin{tabular}{ll}
  \toprule
  sigma & max tau\_l2 \\ 
  \midrule
  0 & 0 \\ 
  0.0491 & 0.0036 \\ 
  0.0982 & 0.0142 \\ 
  0.196 & 0.0535 \\ 
  0.393 & 0.175 \\ 
  \bottomrule
\end{tabular}

\end{minipage}
\end{table}

\begin{figure}[t]
  \centering
  \includegraphics[width=0.9\linewidth]{\figLesRouteMismatch}
  \caption{LES route mismatch in viscous Burgers: RMS $L^2$ norm of $\overline{T_\tau(u)} - T_\tau(\bar u)$ over the spatial grid versus time, for several filter widths $\sigma$; the $\sigma=0$ control yields $\approx 0$.}
  \label{fig:les-route-mismatch}
\end{figure}

\begin{figure}[t]
  \centering
  \includegraphics[width=0.9\linewidth]{\figLesSubgridTerm}
  \caption{Subgrid stress magnitude: RMS $L^2$ norm of $\tau_{\mathrm{sgs}}=\overline{u^2}-(\bar u)^2$ versus time and filter width $\sigma$; the $\sigma=0$ control yields $\approx 0$.}
  \label{fig:les-subgrid-term}
\end{figure}
